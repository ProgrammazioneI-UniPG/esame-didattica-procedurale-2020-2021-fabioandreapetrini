%------------------------- AGGIUNTE --------------------------------------------
\begin{itemize}
    \item [] SOL. ESERCIZIO 1
    \begin{lstlisting}{sol_ex1}
#include <stdio.h> 
  
void fillArray(int *ptrVettore) { 
    ptrVettore[0] = 10; 
} 
  
int main() {
    int vettore[100];  
    fillArray(vettore); 
    printf("%d", vettore[0]); 
    return 0; 
} 
\end{lstlisting}

    \item [] SOL. ESERCIZIO 5 
\begin{lstlisting}{sol_ex5}  
float *buildDoublePointerMatrix(float **arr, int nr, int nc) {
    arr = (float *) malloc(nr * sizeof(float));
    for (int i = 0; i < nr; i++) {
        arr[i] = malloc(nc * sizeof(float));
    }

    int count = 0;
    for (int i = 0; i < nr; i++) {
        for (int j = 0; j < nc; j++) {
            arr[i][j] = (float) count++/125;
        }
    }

    return arr;
}
\end{lstlisting}
    
    \item [] SOL. ESERCIZIO 6 
\begin{lstlisting}{sol_ex6}  
struct cellula {
    int numeroVicini;
    int vivo;
};

struct cellula transizione(struct cellula c) {
    if (c.vivo && (c.numeroVicini < 2 || c.numeroVicini > 3)) {
        c.vivo = 0;
    } 

    return c;
}
\end{lstlisting}


    \item [] SOL. ESERCIZIO 11 
\begin{lstlisting}{sol_ex11}  
#include <stdio.h>
int main(void) {
    char s[] = "stampa la stringa";
    for (char *ptr = s; *ptr != '\0'; ++ptr) {
        printf("%c", *ptr);
    }
}
\end{lstlisting}

\vspace{1.4cm}
    \item [] SOL. ESERCIZIO 12 
\begin{lstlisting}{sol_ex12}  
struct Node* buildList(int k, struct Node* head) {
    struct Node* tail = NULL; 
    struct Node* tmp = NULL; 
    int inputValue;

    printf("Creazione linked list con %d elementi\n", k);

    for (int i = 0; i < k; i++) {
        printf("Inserisci l'elmento %d: ", i);
        scanf("%d", &inputValue);

        if (i == 0) {
            head = (struct Node*) malloc(sizeof(struct Node)); 
            head->data = inputValue;
            head->next = NULL;
            
            tail = head;
        } else {
            tmp = (struct Node*) malloc(sizeof(struct Node)); 
            tmp->data = inputValue;
            tmp->next = NULL;

            tail->next = tmp;
            tail = tmp;
        }
    }

    return head;
}
\end{lstlisting}

    \item [] SOL. ESERCIZIO 15 
\begin{lstlisting}{sol_ex15}  
int elimina(int key, struct Node **radice){
   while( (*radice != NULL) && ((*radice)->data < key)){
      radice = &( (*radice)->next );
   }
   if ((*radice != NULL) && ((*radice)->data == key)){  
      struct Node *next;
      next=(*radice)->next;
      free(*radice);
      *radice=next;
      return(1);
   } else {
      return(0);
   }
}
\end{lstlisting}


\end{itemize}



